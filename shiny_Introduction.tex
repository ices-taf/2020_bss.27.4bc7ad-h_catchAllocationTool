% Options for packages loaded elsewhere
\PassOptionsToPackage{unicode}{hyperref}
\PassOptionsToPackage{hyphens}{url}
%
\documentclass[
]{article}
\usepackage{lmodern}
\usepackage{amssymb,amsmath}
\usepackage{ifxetex,ifluatex}
\ifnum 0\ifxetex 1\fi\ifluatex 1\fi=0 % if pdftex
  \usepackage[T1]{fontenc}
  \usepackage[utf8]{inputenc}
  \usepackage{textcomp} % provide euro and other symbols
\else % if luatex or xetex
  \usepackage{unicode-math}
  \defaultfontfeatures{Scale=MatchLowercase}
  \defaultfontfeatures[\rmfamily]{Ligatures=TeX,Scale=1}
\fi
% Use upquote if available, for straight quotes in verbatim environments
\IfFileExists{upquote.sty}{\usepackage{upquote}}{}
\IfFileExists{microtype.sty}{% use microtype if available
  \usepackage[]{microtype}
  \UseMicrotypeSet[protrusion]{basicmath} % disable protrusion for tt fonts
}{}
\makeatletter
\@ifundefined{KOMAClassName}{% if non-KOMA class
  \IfFileExists{parskip.sty}{%
    \usepackage{parskip}
  }{% else
    \setlength{\parindent}{0pt}
    \setlength{\parskip}{6pt plus 2pt minus 1pt}}
}{% if KOMA class
  \KOMAoptions{parskip=half}}
\makeatother
\usepackage{xcolor}
\IfFileExists{xurl.sty}{\usepackage{xurl}}{} % add URL line breaks if available
\IfFileExists{bookmark.sty}{\usepackage{bookmark}}{\usepackage{hyperref}}
\hypersetup{
  pdftitle={Introduction},
  hidelinks,
  pdfcreator={LaTeX via pandoc}}
\urlstyle{same} % disable monospaced font for URLs
\usepackage[margin=1in]{geometry}
\usepackage{graphicx,grffile}
\makeatletter
\def\maxwidth{\ifdim\Gin@nat@width>\linewidth\linewidth\else\Gin@nat@width\fi}
\def\maxheight{\ifdim\Gin@nat@height>\textheight\textheight\else\Gin@nat@height\fi}
\makeatother
% Scale images if necessary, so that they will not overflow the page
% margins by default, and it is still possible to overwrite the defaults
% using explicit options in \includegraphics[width, height, ...]{}
\setkeys{Gin}{width=\maxwidth,height=\maxheight,keepaspectratio}
% Set default figure placement to htbp
\makeatletter
\def\fps@figure{htbp}
\makeatother
\setlength{\emergencystretch}{3em} % prevent overfull lines
\providecommand{\tightlist}{%
  \setlength{\itemsep}{0pt}\setlength{\parskip}{0pt}}
\setcounter{secnumdepth}{-\maxdimen} % remove section numbering

\title{Introduction}
\author{}
\date{\vspace{-2.5em}}

\begin{document}
\maketitle

\hypertarget{welcome-to-the-seabass-tool}{%
\subsection{Welcome to the Seabass
Tool}\label{welcome-to-the-seabass-tool}}

\begin{center}\rule{0.5\linewidth}{0.5pt}\end{center}

This tool was developed as a technical service to DGMARE in 2019 and
reflect the most recent available data on seabass (Dicentrarchus labrax)
catches in divisions 4.b--c, 7.a, and 7.d--h (central and southern North
Sea, Irish Sea, English Channel, Bristol Channel, and Celtic Sea).

\textbf{The Seabass tool is not an ICES advice and cannot be applied as
such}; the ICES advice on Seabass can be found here:
\href{http://ices.dk/sites/pub/Publication\%20Reports/Advice/2020/2020/bss.27.4bc7ad-h.pdf}{ICES,
2020a}

The sea bass catch allocation tool was developed to be used
\textbf{exclusively} for sea bass (\emph{Dicentrarchus labrax}) in
divisions 4.b--c, 7.a, and 7.d--h (central and southern North Sea, Irish
Sea, English Channel, Bristol Channel, and Celtic Sea) in 2021. The
assumptions used in the tool are closely linked to the assessment of
this stock (ICES, 2020a) and any attempt to use it for other stocks will
not produce sensible results. The Seabass tool is updated annually with
updated data produced by the Working Group for the Celtic Seas Ecoregion
(\href{https://www.ices.dk/community/groups/Pages/WGCSE.aspx}{WGCSE})

The sea bass catch allocation tool was developed with the intent to aid
managers and stakeholders to test multiple catch allocation schemes by
gear and different regulations for recreational catches, using the ICES
catch advice (ICES, 2020b) as a maximum.

Although this tool is designed to be in line with the assumptions used
in the sea bass advice there are instances where it deviates from them.
The tool can be used to allocate monthly options, while the assessment
is run an annual time step. Also, depending on the gear options
selected, the overall commercial catch- at-age may differ from that
assumed in the ICES forecast.

\hypertarget{references}{%
\subsubsection{\texorpdfstring{\textbf{References:}}{References:}}\label{references}}

\begin{center}\rule{0.5\linewidth}{0.5pt}\end{center}

ICES.2020a. Working Group for the Celtic Seas Ecoregion (WGCSE). ICES
Scientific Reports. 2:40. xx
pp.~\url{http://doi.org/10.17895/ices.pub.5978}

ICES.2020b. Sea bass(Dicentrarchus labrax) in divisions 4.b--c,7.a, and
7.d--h (central and southern North Sea, Irish Sea, English Channel,
Bristol Channel, and Celtic Sea. In Report of the ICES Advisory
Committee, 2020. ICES Advice 2020, bss.27.4bc7ad-h.
\url{https://doi.org/10.17895/ices.advice.5916}

\end{document}
